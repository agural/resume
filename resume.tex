\documentclass{article}
\usepackage{fullpage}
\usepackage{amsmath}
\usepackage{amssymb}
\usepackage[usenames]{color}
\usepackage{enumitem}
\usepackage{fixltx2e}
\usepackage{bm}
\usepackage[super]{nth}

\thispagestyle{empty}
\pagestyle{empty}

\leftmargin=0.25in
\oddsidemargin=0.25in
\textwidth=6.0in
\topmargin=-.75in
\textheight=10in

\raggedright
%\renewcommand{\rmdefault}{phv}

\pagenumbering{arabic}

\def\bull{\vrule height .8ex width .7ex depth -.1ex }


% DEFINITIONS FOR RESUME

\newenvironment{changemargin}[2]{%
  \begin{list}{}{%
    \setlength{\topsep}{0pt}%
    \setlength{\leftmargin}{#1}%
    \setlength{\rightmargin}{#2}%
    \setlength{\listparindent}{\parindent}%
    \setlength{\itemindent}{\parindent}%
    \setlength{\parsep}{\parskip}%
  }%
  \item[]}{\end{list}
}

\newcommand{\lineover}{
	\begin{changemargin}{-0.05in}{-0.05in}
		\vspace*{-8pt}
		\hrulefill \\
		\vspace*{-2pt}
	\end{changemargin}
}

\newcommand{\header}[1]{
	\begin{changemargin}{-.5in}{-0.5in}
		{\large \textbf{\scshape{#1}}}\\
  	\lineover
	\end{changemargin}
}

\newcommand{\contact}[6]{
	\begin{changemargin}{1in}{1in}
		\begin{center}
			{\LARGE \scshape {#1}}\\ \smallskip
			{#4}\\ \smallskip
			{#2} \hfill {#5}\\ \smallskip
			{#3} \hfill {#6}\\ \smallskip 
		\end{center}
	\end{changemargin}
}

\newenvironment{body} {
	\vspace*{-16pt}
	\begin{changemargin}{-0.25in}{-0.5in}
  }	
	{\end{changemargin}
}	

\newcommand{\school}[4]{
	\textbf{#1} \hfill \emph{#2\\}
	#3\\ 
	#4\\
}

\newcommand{\CC}{C\nolinebreak\hspace{-.05em}\raisebox{.4ex}{\tiny\bf +}\nolinebreak\hspace{-.10em}\raisebox{.4ex}{\tiny\bf +}}
\newcommand{\rr}{\textsubscript{\textregistered}}

% END RESUME DEFINITIONS

\begin{document}

%%%%%%%%%%%%%%%%%%%%%%%%%%%%%%%%%%%%%%%%%%%%%%%%%%%%%%%%%%%%%%%%%%%%%%%%%%%%%%%%
% Name
\begin{changemargin}{0.5in}{0.5in}
	\begin{center}
		\begin{tabular}{ccc}
			\parbox[t]{3cm}{\hspace{1cm}} &
			\parbox[t]{5.7cm}{\center{{\huge \scshape {Albert Gural}} \\ \smallskip {http://www.albertgural.com/}}} &
			\parbox[t]{6cm}{\center{\vspace{0.45cm} {\small agural@stanford.edu $|$ 703.346.2869}\\
			%{\footnotesize ***Update with Stanford address***}
	}}
		\end{tabular}
	\end{center}
\end{changemargin}

%\contact{Albert Gural}{MSC \#466}{Pasadena, CA 91126-0466}{http://www.albertgural.com/}{agural@caltech.edu}{(703) 346-2869}


%%%%%%%%%%%%%%%%%%%%%%%%%%%%%%%%%%%%%%%%%%%%%%%%%%%%%%%%%%%%%%%%%%%%%%%%%%%%%%%%
% Education
\header{Education}

\begin{body}
	\vspace{14pt}
	\textbf{Stanford University} \hfill \textbf{Stanford, CA} \\
	\emph{Ph.D. Electrical Engineering}, GPA: $4.0$ \hfill \emph{Sept. 2016 - present} \\
	\begin{itemize}
	\item \textbf{Murmann Mixed-Signal Group (Boris Murmann)} \\
	\item \textbf{Selected Coursework:} Fundamentals of Analog IC Design, Advanced Analog IC Design, RF IC Design, Principles and Models of Semiconductor Devices, Digital Signal Processing
	%\item \textbf{Research:}
	\item \textbf{Activities/Awards:} The Krishna Kolluri Graduate Fellowship Fund
	\end{itemize}

	\smallskip

	\textbf{California Institute of Technology} \hfill \textbf{Pasadena, CA} \\
	\emph{B.S. Electrical Engineering with a minor in Computer Science}, GPA: $4.0$ \hfill \emph{June 2016} \\
	\begin{itemize}
	%\item \textbf{Current Coursework:} Mixed-mode Integrated Circuits Research (with Professor Emami)
	\item \textbf{Selected Coursework:} Senior Thesis (MICS Lab), Machine Learning and Data Mining, Advanced Digital Systems, Mixed-mode ICs, Feedback and Control Circuits, Signal-Processing Systems
	\item \textbf{Activities/Awards:} ACM-ICPC (international collegiate programming contest - \textbf{Honorable Mention} at internationals, \textit{2016, Thailand}; $\boldsymbol{1}^\text{\textbf{st}}$ \textbf{place team} at regionals, \textit{2015}; $4^\text{th}$ place team, \textit{2014}),\\The Kiyo and Eiko Tomiyasu SURF Scholar award (Caltech, \textit{2015}), member of Tau Beta Pi honor society
	\end{itemize}

	%\smallskip

	%\textbf{Thomas Jefferson High School for Science and Technology} \hfill \textbf{Alexandria, VA} \\
	%\emph{Senior Research in Computer Science}, GPA: $4.45$ \hfill \emph{Sept. 2008 - June 2012} \\
	%\begin{itemize}
	%\item \textbf{Activities/Awards:} Computer Team (co-captain, \textit{2010-12}), USACO (contest programming, Platinum), ACSL (CS topics, \nth{1} place individual, \textit{2010-11}), Naval Research Lab (\nth{1} place project in CS, \textit{2011})
	%\end{itemize}
\end{body}

\smallskip


%%%%%%%%%%%%%%%%%%%%%%%%%%%%%%%%%%%%%%%%%%%%%%%%%%%%%%%%%%%%%%%%%%%%%%%%%%%%%%%%
% Experience
\header{Work and Experience}

\begin{body}
	\vspace{14pt}
	
	\textbf{Stanford (Boris Murmann Mixed-Signal Group)}, \emph{PhD Research} \hfill \emph{2016-now}\\
	\vspace*{-4pt}
	\begin{itemize}
	    \item Researching methods to lower power consumption of PPG heart rate sensors.
	    \item Preliminary results show machine learning inspired techniques can robustly track heart rate with $50\times$ power reduction over state-of-the-art methods.
	    %\item Analyzing techniques such as compressive sensing to lower power, spectrum track finding for robust heart rate detection, and machine learning classification algorithms for detecting rest conditions.
	\end{itemize}
	
	\smallskip
	
	\textbf{Xilinx}, \emph{Machine Learning Intern} \hfill \emph{Summer 2017}\\
	\vspace*{-4pt}
	\begin{itemize}
		\item Completed two projects - (1) improved Winograd transforms ($100\times$ condition number improvement over state-of-the-art) for convolutional neural network compression; (2) implemented a structured matrices transform technique ($1000\times$ parameter reduction typical) for fully-connected neural network compression.
		\item Used Python, NumPy, and TensorFlow.
	\end{itemize}

	\smallskip

	\textbf{Caltech (Azita Emami MICS Lab)}, \emph{Senior Thesis} \hfill \emph{2015-16}\\
	\vspace*{-4pt}
	\begin{itemize}
		\item VLSI 2017 conference publication. DOI: 10.23919/VLSIC.2017.8008566
		\item Designed a novel low-power, high-linearity PLL-based potentiostat for measuring blood glucose levels.
		\item Fabricated in TSMC 65nm and successfully tested with glucose solutions \emph{in vitro}.
		\item Developed an FPGA/NIOS-II testing framework that lead to huge productivity improvements.
	\end{itemize}

	\smallskip

	\textbf{Caltech (Azita Emami MICS Lab)}, \emph{Named Summer Undergraduate Research Fellow} \hfill \emph{Summer 2015}\\
	\vspace*{-4pt}
	\begin{itemize}
		\item Designed and simulated a novel PLL-based potentiostat for measuring dopamine concentrations \emph{in vivo}.
		\item Used Cadence Virtuoso with 45nm CMOS predictive models. % Insert paper link
	\end{itemize}

	\smallskip

	\textbf{Jane Street Capital}, \emph{Software Developer Intern} \hfill \emph{Summer 2014}\\
	\vspace*{-4pt}
	\begin{itemize}
		\item Completed two projects - (1) fault-tolerant distributed lock server to replace NFS locks; (2) plugin support for the internal trader tool as well as a plugin ecosystem for trader developers with version control.
		\item Used OCaml (including the Async monad, RPCs, DynLoader).
	\end{itemize}

	\smallskip

	\textbf{Google (Research Division)}, \emph{Software Engineering Intern} \hfill \emph{Summer 2013}\\
	\vspace*{-4pt}
	\begin{itemize}
		\item Developed image processing techniques to clean a sequence of object photos to QA specifications, allowing for a much larger class of object image sequences to be processed; currently for Google Shopping.
		\item Used \CC, OpenCV, and the Ceres non-linear solver library.
	\end{itemize}

	%\smallskip

	%\textbf{Naval Research Laboratory}, \emph{Intern, High Performance Computing} \hfill \emph{Summer 2011, 2012}\\
	%\vspace*{-4pt}
	%\begin{itemize}
	%	\item \emph{Summer 2012:} Built a molecular dynamics simulation in C; compared different integration step algorithms including brute force, linked cell, and monotonic Lagrangian grid.
	%	\item \emph{Summer 2011:} Created an MPI (Message Passing Interface) library for parallel operations on a grid in \CC, tested on a wave propagation simulation, then analyzed its efficiency.
	%\end{itemize}

	\smallskip

	\textbf{California Institute of Technology}, \emph{Teaching Assistant} \hfill \emph{2014, 2015} \\
	%\emph{CS 38 (Algorithms; Spring 2014, Spring 2015) and EE 45 (Electronics Laboratory; Winter 2015)}
	\vspace*{-4pt}
	\begin{itemize}
		\item \emph{Algorithms:} Lectured and created course materials for topics including graph algorithms, greedy algorithms, dynamic programming, flow networks, and linear programming.
		\item \emph{Electronics Laboratory:} Conducted homework and laboratory sessions in topics including discrete analog components, op-amp circuits, and differential amplifier circuits.
		%\item Conducted weekly office hours; provided intuition for problem solving as well as concrete examples and big picture overviews. Graded assignments. Received extremely positive reviews from students.
	\end{itemize}
\end{body}

\smallskip

%%%%%%%%%%%%%%%%%%%%%%%%%%%%%%%%%%%%%%%%%%%%%%%%%%%%%%%%%%%%%%%%%%%%%%%%%%%%%%%%
% Projects
\newpage
\header{Projects}
\begin{body}
	\vspace{14pt}
	%\textbf{0.45kW Coilgun - 14g projectile, 40m/s exit velocity, 2\% efficiency}, \emph{group project} \hfill \emph{Summer 2016}\\
	%\textbf{Ray tracer and lens optimization tool using stochastic approximation} \hfill \emph{Summer 2016}\\
	\textbf{Schematic and layout of implantable Potentiostat, fabricated in TSMC 65nm} \hfill \emph{Spring 2016}\\
	\textbf{Design and construction of reflow oven utilizing a fully-analog PI-controller} \hfill \emph{Spring 2016}\\
	\textbf{Potentiostat utilizing an all-digital phase-locked loop in 45nm CMOS technology} \hfill \emph{Summer 2015}\\
	\textbf{Design and construction of 1kW Solid-state Tesla coil} \hfill \emph{Summer 2015} \\
	%\textbf{Electronic automatic dog food dispenser} \hfill \emph{Spring 2015} \\
	\textbf{6-8 GHz all-digital delay-locked loop in 45nm CMOS technology}, \emph{group project} \hfill \emph{Spring 2015}\\
	\textbf{8-bit AVR-compatible processor in VHDL for a Xilinx FPGA}, \emph{group project} \hfill \emph{Winter 2015}\\
	%\textbf{Differential geometry algorithms: mesh smoothing, flattening, surface flow} \hfill \emph{Fall 2014}\\
	%\textbf{3-stage BJT amplifier with flat 40dB gain from 10Hz to 200kHz}, \emph{group project} \hfill \emph{Spring 2014}\\
	\textbf{5MHz bandwidth FPGA-based oscilloscope, designed and built from scratch} \hfill \emph{Spring 2014}\\
	\textbf{Robotrike firmware (interrupt-based OS written exclusively in x86 assembly)} \hfill \emph{Fall 2013}\\
\end{body}

\medskip

%%%%%%%%%%%%%%%%%%%%%%%%%%%%%%%%%%%%%%%%%%%%%%%%%%%%%%%%%%%%%%%%%%%%%%%%%%%%%%%%
% Skills
\header{Tools and Languages}

\begin{body}
	\vspace{14pt}
	C, \CC, Python/NumPy/scikit-learn/TensorFlow, VHDL/Verilog, Java, OCaml, Haskell, x86 Assembly, \LaTeX, Mathematica, Git, Bash, OpenCV, MPI (parallelization platform), Altium, Altera and Xilinx toolchains, SPICE, Cadence Virtuoso
\end{body}

%\begin{body}
	%\vspace{14pt}
	%\textbf{Tools and Languages:} \\
	%C/\CC, Java, Python, OCaml, Haskell, VHDL/Verilog, x86 Assembly, \LaTeX, Mathematica, Git, Bash, OpenCV, MPI (parallelization platform), Altium\rr, Altera\rr and Xilinx\rr toolchains, Cadence\rr Virtuoso\rr, etc. \\
	%
	%\medskip
	%
	%\textbf{Hobbies and Interests:} Mathematics and Computer Science; Analog and Digital Electronics; Puzzles; Designing, Developing, and Constructing Electronic Devices (mix of EE, CS, and ME). \\
	%% (uncomment when updated) See website for detailed project descriptions (http://www.albertgural.com/projects/). \\
%\end{body}

\end{document}
