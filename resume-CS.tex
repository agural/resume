\documentclass{article}
\usepackage{fullpage}
\usepackage{amsmath}
\usepackage{amssymb}
\usepackage[usenames]{color}
\usepackage{enumitem}
\usepackage[super]{nth}

\thispagestyle{empty}
\pagestyle{empty}

\leftmargin=0.25in
\oddsidemargin=0.25in
\textwidth=6.0in
\topmargin=-0.25in
\textheight=9.25in

\raggedright
%\renewcommand{\rmdefault}{phv}

\pagenumbering{arabic}

\def\bull{\vrule height .8ex width .7ex depth -.1ex }


% DEFINITIONS FOR RESUME

\newenvironment{changemargin}[2]{%
  \begin{list}{}{%
    \setlength{\topsep}{0pt}%
    \setlength{\leftmargin}{#1}%
    \setlength{\rightmargin}{#2}%
    \setlength{\listparindent}{\parindent}%
    \setlength{\itemindent}{\parindent}%
    \setlength{\parsep}{\parskip}%
  }%
  \item[]}{\end{list}
}

\newcommand{\lineover}{
	\begin{changemargin}{-0.05in}{-0.05in}
		\vspace*{-8pt}
		\hrulefill \\
		\vspace*{-2pt}
	\end{changemargin}
}

\newcommand{\header}[1]{
	\begin{changemargin}{-0.5in}{-0.5in}
		{\large \textbf{\scshape{#1}}}\\
  	\lineover
	\end{changemargin}
}

\newcommand{\contact}[6]{
	\begin{changemargin}{1in}{1in}
		\begin{center}
			{\LARGE \scshape {#1}}\\ \smallskip
			{#4}\\ \smallskip
			{#2} \hfill {#5}\\ \smallskip
			{#3} \hfill {#6}\\ \smallskip 
		\end{center}
	\end{changemargin}
}

\newenvironment{body} {
	\vspace*{-16pt}
	\begin{changemargin}{-0.25in}{-0.5in}
  }	
	{\end{changemargin}
}	

\newcommand{\school}[4]{
	\textbf{#1} \hfill \emph{#2\\}
	#3\\ 
	#4\\
}

\newcommand{\CC}{C\nolinebreak\hspace{-.05em}\raisebox{.4ex}{\tiny\bf +}\nolinebreak\hspace{-.10em}\raisebox{.4ex}{\tiny\bf +}}

% END RESUME DEFINITIONS

\begin{document}

%%%%%%%%%%%%%%%%%%%%%%%%%%%%%%%%%%%%%%%%%%%%%%%%%%%%%%%%%%%%%%%%%%%%%%%%%%%%%%%%
% Name
\begin{changemargin}{0.5in}{0.5in}
	\begin{center}
		\begin{tabular}{ccc}
			\parbox[t]{3cm}{\hspace{1cm}} &
			\parbox[t]{5.7cm}{\center{{\huge \scshape {Albert Gural}}\\ \smallskip {http://www.albertgural.com/}}} &
			\parbox[t]{6cm}{\center{\vspace{0.45cm}{\small agural@caltech.edu $|$ 703.346.2869}\\ {\footnotesize MSC \#466, Pasadena, CA 91126-0466}}}
		\end{tabular}
	\end{center}
\end{changemargin}

%\contact{Albert Gural}{MSC \#466}{Pasadena, CA 91126-0466}{http://www.albertgural.com/}{agural@caltech.edu}{(703) 346-2869}


%%%%%%%%%%%%%%%%%%%%%%%%%%%%%%%%%%%%%%%%%%%%%%%%%%%%%%%%%%%%%%%%%%%%%%%%%%%%%%%%
% Education
\header{Education}

\begin{body}
	\vspace{14pt}
	\textbf{California Institute of Technology} \hfill \textbf{Pasadena, CA} \\
	\emph{Electrical Engineering with a minor in Computer Science}, GPA: $4.0/4.3$ \hfill \emph{Oct. 2012 - present} \\
	\begin{itemize}
	\item \textbf{Current Coursework:} Quantum Computation, Advanced Digital Systems Design, Machine Learning and Data Mining, Mixed Mode Integrated Circuits Research (with Professor Emami)
	\item \textbf{Past Coursework:} Algorithms, Computing Systems, Embedded Systems (FPGA Oscilloscope), Feedback and Control Circuits, Signal-Processing Systems, Semiconductor Devices, Abstract Algebra, Combinatorial Analysis, Stochastic Processes, Discrete Differential Geometry
	\item \textbf{Activities:} ACM-ICPC Programming Contest (\textit{2012-14}), Robotics Team (Electronics) (\textit{2012-14})
	\item \textbf{Awards:} ACM-ICPC ($3^\text{rd}$ place at regionals, \textit{2013}; $4^\text{th}$ place at regionals, \textit{2014})
	\end{itemize}

	\medskip

	\textbf{Thomas Jefferson High School for Science and Technology} \hfill \textbf{Alexandria, VA} \\
	\emph{Senior Research in Computer Science}, GPA: $4.45/4.00$ \hfill \emph{Sept. 2008 - June 2012} \\
	\begin{itemize}
	\item \textbf{Relevant Coursework:} Microprocessor Electronics, Artificial Intelligence, Single and Multivariable Calculus, Advanced Math Techniques (Linear, Integration, Series Expansions, Distributions, etc.)
	\item \textbf{Activities:} Computer Team (co-captain, \textit{2010-12}), Varsity Math Team, Botball Robotics, Physics Team
	\item \textbf{Awards:} USACO (algorithmic coding, gold division), ACSL (CS topics, \nth{1} place individual, \textit{2010-11}), AIME qualifier (math, \textit{2009-12}), Naval Research Lab (\nth{1} place project in CS, \textit{2011})
	\end{itemize}
\end{body}

\smallskip


%%%%%%%%%%%%%%%%%%%%%%%%%%%%%%%%%%%%%%%%%%%%%%%%%%%%%%%%%%%%%%%%%%%%%%%%%%%%%%%%
% Experience
\header{Work and Experience}

\begin{body}
	\vspace{14pt}
	\textbf{California Institute of Technology}, \emph{Teaching Assistant} \hfill \emph{Spring 2014, Winter 2015}\\
	\emph{CS 38 (Algorithms; Spring 2014) and EE 45 (Electronics Laboratory; Winter 2015)}
	\vspace*{-4pt}
	\begin{itemize}
		\item Conducted weekly office hours; provided intuition for problem solving as well as concrete examples and big picture overviews. Graded assignments. 
		\item Received $100\%$ positive reviews (many commented on my ability to explain difficult material well).
	\end{itemize}

	\medskip

	\textbf{Jane Street Capital}, \emph{Software Developer Intern} \hfill \emph{Summer 2014}\\
	\vspace*{-4pt}
	\begin{itemize}
		\item Completed two projects - (1) fault-tolerant distributed lock server to replace NFS locks; (2) plugin support for the internal trader tool as well as a plugin ecosystem for trader developers with version control.
		\item Used OCaml (especially the Async monad, RPCs, DynLoader).
	\end{itemize}

	\medskip

	\textbf{Google (Research Division)}, \emph{Software Engineering Intern} \hfill \emph{Summer 2013}\\
	\vspace*{-4pt}
	\begin{itemize}
		\item Developed image processing techniques to clean a sequence of object photos to QA specifications, allowing for a much larger class of object image sequences to be processed; currently for Google Shopping.
		\item Used \CC, OpenCV, and the Ceres non-linear solver library.
	\end{itemize}

	\medskip

	\textbf{Naval Research Laboratory}, \emph{Intern, High Performance Computing} \hfill \emph{Summer 2011, 2012}\\
	\vspace*{-4pt}
	\begin{itemize}
		\item \emph{Summer 2012:} Built a molecular dynamics simulation in C; compared different integration step algorithms including brute force, linked cell, and monotonic Lagrangian grid.
		\item \emph{Summer 2011:} Created an MPI (Message Passing Interface) library for parallel operations on a grid in \CC, tested on a wave propagation simulation, then analyzed its efficiency.
	\end{itemize}
\end{body}

\smallskip


%%%%%%%%%%%%%%%%%%%%%%%%%%%%%%%%%%%%%%%%%%%%%%%%%%%%%%%%%%%%%%%%%%%%%%%%%%%%%%%%
% Skills
\header{Skills and Interests}

\begin{body}
	\vspace{14pt}
	\textbf{Tools and Languages:} \\
	C/\CC, Java, OCaml, VHDL, x86 Assembly, \LaTeX, Mathematica, Git, some Bash, OpenCV, MPI (parallelization platform), Altium, Altera and Xilinx toolchains, many other proprietary tools \\
	
	\medskip
	
	\textbf{Hobbies and Interests:} Mathematics and Computer Science; Analog and Digital Electronics; Puzzles; Designing, Developing, and Constructing Electronic Devices (mix of EE, CS, and ME) \\ See website for detailed project descriptions (http://www.albertgural.com/projects/). \\
\end{body}

\smallskip

\end{document}